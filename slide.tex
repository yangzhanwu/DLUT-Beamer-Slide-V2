\documentclass{beamer}
\usepackage{ctex, hyperref}
\usepackage[T1]{fontenc}

% other packages
\usepackage{latexsym,amsmath,xcolor,multicol,booktabs,calligra}
\usepackage{graphicx,pstricks,listings,stackengine}

\author{杨占武}
\setbeamerfont{title}{size=\large}
\title{LD-LED融合的水下无线光通信MAC协议研究}
% \subtitle{毕业设计答辩}
\institute{大连理工大学软件学院}
\date{2024年5月17日}
\usepackage{dlut}

% defs
\def\cmd#1{\texttt{\color{red}\footnotesize $\backslash$#1}}
\def\env#1{\texttt{\color{blue}\footnotesize #1}}
\definecolor{deepblue}{rgb}{0,0,0.5}
\definecolor{deepred}{rgb}{0.6,0,0}
\definecolor{deepgreen}{rgb}{0,0.5,0}
\definecolor{halfgray}{gray}{0.55}

\lstset{
    basicstyle=\ttfamily\small,
    keywordstyle=\bfseries\color{deepblue},
    emphstyle=\ttfamily\color{deepred},    % Custom highlighting style
    stringstyle=\color{deepgreen},
    numbers=left,
    numberstyle=\small\color{halfgray},
    rulesepcolor=\color{red!20!green!20!blue!20},
    frame=shadowbox,
}


\begin{document}

\kaishu
\begin{frame}
	\titlepage
	\begin{figure}[htpb]
		\begin{center}
			\includegraphics[width=0.2\linewidth]{pic/DLUT-logo.eps}
		\end{center}
	\end{figure}
\end{frame}

\begin{frame}
	\tableofcontents[sectionstyle=show,subsectionstyle=show/shaded/hide,subsubsectionstyle=show/shaded/hide]
\end{frame}


\section{绪论}

\begin{frame}{用Beamer很高大上?}
	\begin{itemize}[<+-| alert@+>] % 当然,除了alert,手动在里面插 \pause 也行
		\item 大家都会\LaTeX{},好多学校都有自己的Beamer主题
		\item 中文支持请选择 Xe\LaTeX{} 编译选项
		\item GitHub项目地址位于 \url{https://github.com/fuujiro/dlut-Beamer-Slide},如果有bug或者feature request可以去里面提issue
	\end{itemize}
\end{frame}


\section{水下无线光通信物理层特性及经典 MAC 协议}

\subsection{Beamer主题分类}

\begin{frame}
	\begin{itemize}
		\item 有一些 \LaTeX{} 自带的
		\item 有一些dlut的
		\item 本模板来源自 \newline \url{https://www.latexstudio.net/archives/4051.html}
		\item 但是最初的 \href{http://far.tooold.cn/post/latex/beamerdlut}{\color{purple}{link}} \cite{origin}已经失效了
		\item 整体设计参考自[Trinkle23897 / THU-Beamer-Theme](https://github.com/Trinkle23897/THU-Beamer-Theme)
	\end{itemize}
\end{frame}


\section{LD LED 融合的水下无线光通信系统及其 MAC 协议设计}

\subsection{美化主题}



\subsection{如何更好地做Beamer}

\begin{frame}{Why Beamer}
	\begin{itemize}
		\item \LaTeX 广泛用于学术界,期刊会议论文模板
	\end{itemize}
	\begin{table}[h]
		\centering
		\begin{tabular}{c|c}
			Microsoft\textsuperscript{\textregistered}  Word & \LaTeX        \\
			\hline
			文字处理工具                                           & 专业排版软件        \\
			容易上手,简单直观                                        & 容易上手          \\
			所见即所得                                            & 所见即所想,所想即所得   \\
			高级功能不易掌握                                         & 进阶难,但一般用不到    \\
			处理长文档需要丰富经验                                      & 和短文档处理基本无异    \\
			花费大量时间调格式                                        & 无需担心格式,专心作者内容 \\
			公式排版差强人意                                         & 尤其擅长公式排版      \\
			二进制格式,兼容性差                                       & 文本文件,易读、稳定    \\
			付费商业许可                                           & 自由免费使用        \\
		\end{tabular}
	\end{table}
\end{frame}

\begin{frame}{排版举例}
	\begin{exampleblock}{无编号公式} % 加 * 
		\begin{equation*}
			J(\theta) = \mathbb{E}_{\pi_\theta}[G_t] = \sum_{s\in\mathcal{S}} d^\pi (s)V^\pi(s)=\sum_{s\in\mathcal{S}} d^\pi(s)\sum_{a\in\mathcal{A}}\pi_\theta(a|s)Q^\pi(s,a)
		\end{equation*}
	\end{exampleblock}
	\begin{exampleblock}{多行多列公式\footnote{如果公式中有文字出现,请用 $\backslash$mathrm\{\} 或者 $\backslash$text\{\} 包含,不然就会变成 $clip$,在公式里看起来比 $\mathrm{clip}$ 丑非常多。}}
		% 使用 & 分隔
		\begin{align}
			Q_\mathrm{target} & =r+\gamma Q^\pi(s^\prime, \pi_\theta(s^\prime)+\epsilon)  \\
			\epsilon          & \sim\mathrm{clip}(\mathcal{N}(0, \sigma), -c, c)\nonumber
		\end{align}
	\end{exampleblock}
\end{frame}

\begin{frame}
	\begin{exampleblock}{编号多行公式}
		% Taken from Mathmode.tex
		\begin{multline}
			A=\lim_{n\rightarrow\infty}\Delta x\left(a^{2}+\left(a^{2}+2a\Delta x+\left(\Delta x\right)^{2}\right)\right.\label{eq:reset}\\
			+\left(a^{2}+2\cdot2a\Delta x+2^{2}\left(\Delta x\right)^{2}\right)\\
			+\left(a^{2}+2\cdot3a\Delta x+3^{2}\left(\Delta x\right)^{2}\right)\\
			+\ldots\\
			\left.+\left(a^{2}+2\cdot(n-1)a\Delta x+(n-1)^{2}\left(\Delta x\right)^{2}\right)\right)\\
			=\frac{1}{3}\left(b^{3}-a^{3}\right)
		\end{multline}
	\end{exampleblock}
\end{frame}

\begin{frame}{图形与分栏}
	% From thuthesis user guide.
	\begin{minipage}[c]{0.3\linewidth}
		\psset{unit=0.8cm}
		\begin{pspicture}(-1.75,-3)(3.25,4)
			\psline[linewidth=0.25pt](0,0)(0,4)
			\rput[tl]{0}(0.2,2){$\vec e_z$}
			\rput[tr]{0}(-0.9,1.4){$\vec e$}
			\rput[tl]{0}(2.8,-1.1){$\vec C_{ptm{ext}}$}
			\rput[br]{0}(-0.3,2.1){$\theta$}
			\rput{25}(0,0){%
				\psframe[fillstyle=solid,fillcolor=lightgray,linewidth=.8pt](-0.1,-3.2)(0.1,0)}
			\rput{25}(0,0){%
				\psellipse[fillstyle=solid,fillcolor=yellow,linewidth=3pt](0,0)(1.5,0.5)}
			\rput{25}(0,0){%
				\psframe[fillstyle=solid,fillcolor=lightgray,linewidth=.8pt](-0.1,0)(0.1,3.2)}
			\rput{25}(0,0){\psline[linecolor=red,linewidth=1.5pt]{->}(0,0)(0.,2)}
			%           \psRotation{0}(0,3.5){$\dot\phi$}
			%           \psRotation{25}(-1.2,2.6){$\dot\psi$}
			\psline[linecolor=red,linewidth=1.25pt]{->}(0,0)(0,2)
			\psline[linecolor=red,linewidth=1.25pt]{->}(0,0)(3,-1)
			\psline[linecolor=red,linewidth=1.25pt]{->}(0,0)(2.85,-0.95)
			\psarc{->}{2.1}{90}{112.5}
			\rput[bl](.1,.01){C}
		\end{pspicture}
	\end{minipage}\hspace{1cm}
	\begin{minipage}{0.5\linewidth}
		\medskip
		%\hspace{2cm}
		\begin{figure}[h]
			\centering
			\includegraphics[height=.4\textheight]{pic/dtmf.pdf}
		\end{figure}
	\end{minipage}
\end{frame}

\begin{frame}[fragile]{\LaTeX{} 常用命令}
	\begin{exampleblock}{命令}
		\centering
		\footnotesize
		\begin{tabular}{llll}
			\cmd{chapter}   & \cmd{section} & \cmd{subsection} & \cmd{paragraph}       \\
			章               & 节             & 小节               & 带题头段落                 \\\hline
			\cmd{centering} & \cmd{emph}    & \cmd{verb}       & \cmd{url}             \\
			居中对齐            & 强调            & 原样输出             & 超链接                   \\\hline
			\cmd{footnote}  & \cmd{item}    & \cmd{caption}    & \cmd{includegraphics} \\
			脚注              & 列表条目          & 标题               & 插入图片                  \\\hline
			\cmd{label}     & \cmd{cite}    & \cmd{ref}                                \\
			标号              & 引用参考文献        & 引用图表公式等                                  \\\hline
		\end{tabular}
	\end{exampleblock}
	\begin{exampleblock}{环境}
		\centering
		\footnotesize
		\begin{tabular}{lll}
			\env{table}   & \env{figure}    & \env{equation}    \\
			表格            & 图片              & 公式                \\\hline
			\env{itemize} & \env{enumerate} & \env{description} \\
			无编号列表         & 编号列表            & 描述                \\\hline
		\end{tabular}
	\end{exampleblock}
\end{frame}

\begin{frame}[fragile]{\LaTeX{} 环境命令举例}
	\begin{minipage}{0.5\linewidth}
		\begin{lstlisting}[language=TeX]
\begin{itemize}
  \item A \item B
  \item C
  \begin{itemize}
    \item C-1
  \end{itemize}
\end{itemize}
\end{lstlisting}
	\end{minipage}\hspace{1cm}
	\begin{minipage}{0.3\linewidth}
		\begin{itemize}
			\item A
			\item B
			\item C
			      \begin{itemize}
				      \item C-1
			      \end{itemize}
		\end{itemize}
	\end{minipage}
	\medskip
	\pause
	\begin{minipage}{0.5\linewidth}
		\begin{lstlisting}[language=TeX]
\begin{enumerate}
  \item 巨佬 \item 大佬
  \item 萌新
  \begin{itemize}
    \item[n+e] 瑟瑟发抖
  \end{itemize}
\end{enumerate}
\end{lstlisting}
	\end{minipage}\hspace{1cm}
	\begin{minipage}{0.3\linewidth}
		\begin{enumerate}
			\item 巨佬
			\item 大佬
			\item 萌新
			      \begin{itemize}
				      \item[n+e] 瑟瑟发抖
			      \end{itemize}
		\end{enumerate}
	\end{minipage}
\end{frame}

\begin{frame}[fragile]{\LaTeX{} 数学公式}
	\begin{columns}
		\begin{column}{.55\textwidth}
			\begin{lstlisting}[language=TeX]
$V = \frac{4}{3}\pi r^3$

\[
  V = \frac{4}{3}\pi r^3
\]

\begin{equation}
  \label{eq:vsphere}
  V = \frac{4}{3}\pi r^3
\end{equation}
\end{lstlisting}
		\end{column}
		\begin{column}{.4\textwidth}
			$V = \frac{4}{3}\pi r^3$
			\[
				V = \frac{4}{3}\pi r^3
			\]
			\begin{equation}
				\label{eq:vsphere}
				V = \frac{4}{3}\pi r^3
			\end{equation}
		\end{column}
	\end{columns}
	\begin{itemize}
		\item 更多内容请看 \href{https://zh.wikipedia.org/wiki/Help:数学公式}{\color{purple}{这里}}
	\end{itemize}
\end{frame}

\begin{frame}[fragile]
	\begin{columns}
		\column{.6\textwidth}
		\begin{lstlisting}[language=TeX]
    \begin{table}[htbp]
      \caption{编号与含义}
      \label{tab:number}
      \centering
      \begin{tabular}{cl}
        \toprule
        编号 & 含义 \\
        \midrule
        1 & 4.0 \\
        2 & 3.7 \\
        \bottomrule
      \end{tabular}
    \end{table}
    公式~(\ref{eq:vsphere}) 的
    编号与含义请参见
    表~\ref{tab:number}。
\end{lstlisting}
		\column{.4\textwidth}
		\begin{table}[htpb]
			\centering
			\caption{编号与含义}
			\label{tab:number}
			\begin{tabular}{cl}\toprule
				编号 & 含义  \\\midrule
				1  & 4.0 \\
				2  & 3.7 \\\bottomrule
			\end{tabular}
		\end{table}
		\normalsize 公式~(\ref{eq:vsphere})的编号与含义请参见表~\ref{tab:number}。
	\end{columns}
\end{frame}

\begin{frame}{作图}
	\begin{itemize}
		\item 矢量图 eps, ps, pdf
		      \begin{itemize}
			      \item METAPOST, pstricks, pgf $\ldots$
			      \item Xfig, Dia, Visio, Inkscape $\ldots$
			      \item Matlab / Excel 等保存为 pdf
		      \end{itemize}
		\item 标量图 png, jpg, tiff $\ldots$
		      \begin{itemize}
			      \item 提高清晰度,避免发虚
			      \item 应尽量避免使用
		      \end{itemize}
	\end{itemize}
	\begin{figure}[htpb]
		\centering
		\includegraphics[width=0.2\linewidth]{pic/DLUT-logo.eps}
		\caption{这个校徽就是矢量图}
	\end{figure}
\end{frame}


\section{基于 ns-3 的 LD-LED 融合的水下无线光通信系统及 MAC 协议仿真}
\begin{frame}
	\begin{itemize}
		\item 一月:完成文献调研
		\item 二月:复现并评测各种Beamer主题美观程度
		\item 三、四月:美化dlut Beamer主题
		\item 五月:论文撰写
	\end{itemize}
\end{frame}


\section{结论与展望}

\begin{frame}[allowframebreaks]
	\bibliography{ref}
	\bibliographystyle{alpha}
	% 如果参考文献太多的话,可以像下面这样调整字体:
	% \tiny\bibliographystyle{alpha}
\end{frame}

\begin{frame}
	\begin{center}
		{\Huge\calligra Thanks!}
	\end{center}
\end{frame}

\end{document}